%%%%%%%%%%%%%%%%%%%%%%%%%%%%%%%%%%%%%%%%%%%%%%%%%%%%%%%%%%%%%%%%%%%%%%%%%%%%%%%
% 学术简历 LaTeX 模板
%%%%%%%%%%%%%%%%%%%%%%%%%%%%%%%%%%%%%%%%%%%%%%%%%%%%%%%%%%%%%%%%%%%%%%%%%%%%%%%
\documentclass[11pt,a4paper]{article}

%%%%%%%%%%%%%%%%%%%%%%%%%%%%%%%%%%%%%%%%%%%%%%%%%%%%%%%%%%%%%%%%%%%%%%%%%%%%%%%
% 自定义信息
%%%%%%%%%%%%%%%%%%%%%%%%%%%%%%%%%%%%%%%%%%%%%%%%%%%%%%%%%%%%%%%%%%%%%%%%%%%%%%%
% 个人信息
\newcommand{\Title}{学术简历}
\newcommand{\Name}{狄会哲}
\newcommand{\Role}{海洋地质\hspace{0.5em}海洋地球物理\hspace{0.5em}博士研究生}
\newcommand{\Email}{dihuizhe@scsio.ac.cn}
\newcommand{\Website}{+86 18813175977}
\newcommand{\Github}{HuizheDi}
\newcommand{\Orcid}{0000-0003-3059-712X}
\newcommand{\Affiliation}{男, 中共党员, 1995年1月生 \\ 中国科学院南海海洋研究所\\ 边缘海与大洋地质重点实验室}
\newcommand{\Address}{广东省广州市海珠区新港西路 164 号 2号楼 506b 室}

% 在文章列表中引用作者和合作者
\newcommand{\Di}{\textbf{Di, H.}}
\newcommand{\狄}{\textbf{狄会哲}}

% 一些命令
\newcommand{\DOI}[1]{doi:\href{https://dx.doi.org/#1}{#1}}

% 中文支持
\usepackage[fontset=fandol]{ctex}
% 设置页面边距
\usepackage[margin=2.5cm]{geometry}
% 设置字体
\usepackage{fontspec}
\usepackage[default,semibold]{sourcesanspro}

% Use fontawesome icons
\usepackage[fixed]{fontawesome5}

% 对列表各项逆序编号(用于对文章进行编号)
\usepackage[itemsep=2pt]{etaremune}

% 控制文字字号
\usepackage{anyfontsize}

% 以“年/月”格式显示日期
\usepackage{datetime}
\newdateformat{monthyear}{\THEYEAR/\twodigit{\THEMONTH}}

% 设置每节标题的前后空白
\usepackage{titlesec}
\titlespacing*{\section}{0pt}{1ex}{1ex}
% 设置 section, subsection 不显示编号,且可生成目录
\titleformat{\section}{\normalfont\Large\bfseries}{}{0pt}{}
\titleformat{\subsection}{\normalfont\large\bfseries}{}{0.5em}{}

% 设置行间距
\renewcommand{\baselinestretch}{1.2}
% 设置表格的垂直距离
\renewcommand{\arraystretch}{1.2}
\setlength{\parindent}{0pt} % no indent for paragraph

% 设置列表中各项之间的间距
\usepackage{enumitem}
\setlist{itemsep=0pt}

% 长表格
\usepackage{tabularx}
\usepackage{ltablex}
\usepackage{environ}
\NewEnviron{EntriesTable}[3]{
\vspace{-1.25em}
\begin{tabularx}{\textwidth}{p{#1\textwidth}@{\hspace{#2\textwidth}}p{#3\textwidth}}
    \BODY
\end{tabularx}
}

% 获取总页数
\usepackage{lastpage}

% 设置页眉页脚
\usepackage{fancyhdr}
\pagestyle{fancy}
\fancyhf{}
\chead{
    \itshape
    \fontsize{10pt}{12pt}\selectfont
    \Name
    \hspace{0.2cm} -- \hspace{0.2cm}
    \Title
    \hspace{0.2cm} -- \hspace{0.2cm}
    \monthyear\today
}
\rhead{}
\cfoot{\fontsize{10pt}{0}\selectfont \thepage/\pageref*{LastPage}}
\renewcommand{\headrulewidth}{0pt}

% 使用自定义颜色
\usepackage[usenames,dvipsnames]{xcolor}

% PDF 元信息以及超链接
\usepackage[colorlinks=true]{hyperref}
\hypersetup{ % document metadata
    pdftitle = {\Name\ - \Title},
    pdfauthor = {\Name},
    linkcolor=black,
    citecolor=black,
    filecolor=black,
    urlcolor=MidnightBlue,
}
%%%%%%%%%%%%%%%%%%%%%%%%%%%%%%%%%%%%%%%%%%%%%%%%%%%%%%%%%%%%%%%%%%%%%%%%%%%%%%%

\begin{document}

% CV 封面页
\thispagestyle{empty} % 首页不显示页眉页脚
\begin{center}
\kaishu
{\fontsize{24pt}{0}\selectfont \Name \hspace{1ex}} \\[0.4cm]
{\fontsize{16pt}{0}\selectfont \Role} \\[0.2cm]
\end{center}
\begin{minipage}[t]{0.6\textwidth}
  \kaishu
  \fontsize{12pt}{15pt}\selectfont
  \Affiliation
  \\
  \Address
\end{minipage}
\begin{minipage}[t]{0.4\textwidth}
  \kaishu
  \fontsize{12pt}{15pt}\selectfont
  \begin{flushleft}
    \faEnvelope \href{mailto:\Email}{\texttt{\Email}}
	\\
	\faOrcid \href{https://orcid.org/\Orcid}{\Orcid}
	\\
    \faGlobe \href{https://\Website}{\Website}
	\\
	\faGithub \href{https://github.com/\Github}{\Github}
  \end{flushleft}
\end{minipage}
\vspace{0.2cm}

\section{Education}

\begin{EntriesTable}{0.18}{0.02}{0.80}
2018/09-Now & Ph.D in Marine Geology, University of Chinese Academy of Sciences, China \\
2014/09-2018/06 & B.S. in Engineering, Chengdu Univerisity of Technology, China \\
\end{EntriesTable}

\section{Research Interests}

\begin{itemize}
\item Active Seismic Imaging in Oceanic Lithosphere
\item Magmatic-Tectonic-Hydrothermal Interactions at Mid-ocean Ridges
\item Deep Learning in Seismic Imaging
\end{itemize}

\section{技能}

\begin{itemize}
\item 大学英语四级(539)和六级(483)考试
\item 计算机二级证书(C语言程序设计)
\item 多道地震数据处理, 向下延拓, 走时层析成像, 全波形反演, 逆时偏移成像, 热液系统模拟
\item Linux, Matlab, GMT, Python, PyTorch ...
\end{itemize}

\section{Awards \& Honors}

\begin{EntriesTable}{0.05}{0.02}{0.93}
2022 & Merit Student, Chinese Academy of Sciences \\
2021 & Excellent Party Member, Graduate Geology Class Party branch \\
2019 & Merit Student, Chinese Academy of Sciences \\
2018 & Scholarships for Incoming Freshmen, Chinese Academy of Sciences \\
2018 & Outstanding Graduate Student, Chengdu University of Technology \\
2016 & Third prize of "Application of Geological Skills", the 4th CDUT Geological Skills Competition \\
2016 & University-Level Outstanding Student, Chengdu University of Technology \\
2015 & University-Level Learning Excellence Award, Chengdu University of Technology \\
\end{EntriesTable}

\section{已发表论文}
% AGU style: https://publications.agu.org/agu-grammar-and-style-guide/
\newcommand{\CS}{*} % 通讯作者
\newcommand{\CF}{\textsuperscript{\#}} % 共同一作

\CS 通讯作者,\CF 共同一作
\begin{etaremune}
\item
    \狄, 谢文鑫, \& 徐敏. (2022).
    东太平洋北部洋中脊上地壳精细结构地震探测.
    \textit{地球物理学报}, \textit{66}(4), 1618--1633.
    \DOI{10.6038/cjg2022Q0006}.
\item
    Zhang, M., \Di\CF, Xu, M., Canales, C., Yu, C., Zhao, X., Wang, P., Zeng, X., \& Wang, Y. (2022).
    Seismic Imaging of Dante's Domes Oceanic Core Complex From Streamer Waveform Inversion and Reverse Time Migration.
    \textit{Journal of Geophysical Research: Solid Earth}, \textit{127}, e2021JB023814.
    \DOI{10.1029/2021JB023814}.
\item
    张茂传, 徐敏, 赵旭, 张佳政, 查财财, Vithana, M., \狄, \& 曾信. (2020).
    Kirchhoff向下延拓法在海洋合成多道地震走时反演中的应用.
    \textit{热带海洋学报}, \textit{39}(4), 80--90.
    \DOI{10.11978/2019087}.
\item
    徐敏, \狄, 周志远, 李海勇, \& 林间. (2019).
    俯冲带水圈-岩石圈相互作用研究进展与启示.
    \textit{海洋地质与第四纪地质}, \textit{39}(5), 58--70.
    \DOI{10.16562/j.cnki.0256-1492.2019063001}.
\item
    \狄, 邓宾, 赵高平, 叶月豪, \& 邱嘉文. (2018).
    云贵高原河流水系演化与高原形成过程——基于现代河流沉积物示踪.
    \textit{四川地质学报}, \textit{38}(4), 536--541.
    \DOI{10.3969/j.issn.1006-0995.2018.04.002}.
\end{etaremune}

\subsection*{修改/审稿中}
\begin{etaremune}
\item
    \Di, Xu, M., \& Xie, W.
    Upper crustal Vp/Vs ratios along the northern East Pacific Rise derived from downward-continued streamer data.
    \textit{Submitted to Geophysical Journal International}.
\item
    Xie, W., \Di, Zhang, M., \&  Xu, M.
    High-resolution seismic imaging of shallow structure near proposed IODP drilling sites, Kane Oceanic Core Complex, Mid-Atlantic Ridge.
    \textit{Submitted to Solid Earth Sciences}.
\end{etaremune}

%\subsection*{准备中}
%\begin{etaremune}
%\end{etaremune}

\section{会议摘要}

\begin{etaremune}
\item \狄, \& 徐敏. (2022).
    Upper crustal Vp/Vs ratios along the northern East Pacific Rise derived from downward-continued streamer data.
    2021/2022 中国地球科学联合学术年会, 福州, 福建, 海洋地球物理-口头报告
\item \狄, 徐敏, \& 张茂传. (2021).
    海洋多道反射地震成像在洋中脊热液系统成像的应用.
    2021 第六届地球系统科学大会, 上海, 专题32(大洋岩石圈成因与深部物质循环)-口头报告
\item \狄, 徐敏, \& 张茂传. (2020).
    东太平洋海隆8°-10°N三维热液通道地震探测.
    2020 印度洋国际地质研讨会暨边缘海与大洋地质实验室2019年会, 清远, 广东, 展板报告
\item \狄, 徐敏, 张茂传, \& 赵旭. (2019).
    基于向下延拓数据的东太平洋海隆热液通道地震探测.
    2019 海洋地球科学研讨会暨边缘海与大洋地质实验室2018年会, 广州, 广东, 展板报告
\end{etaremune}

\section{Field Experience}
\begin{itemize}
\item Yongxing Island, Sansha City, Hainan Province--Xisha Deep-sea Marine Environment Observation and Research Station,
    2021/10/30 -- 2021/11/05, Recovery and deployment of short period and 4G seismometers
\item Yongxing Island, Sansha City, Hainan Province--Xisha Deep-sea Marine Environment Observation and Research Station,
    2020/11/19 -- 2020/11/27, Deployment of wide band and short period seismometers, DAS
\item South China Sea -- Geological and ecological Shared Voyage 2019 by the Center for Integrated Island and Reef Research, Chinese Academy of Sciences, "Shiyan 2",
    2019/07/10 -- 2019/08/27, Multichannel seismic and ocean bottom seismometers
\end{itemize}




\end{document}
